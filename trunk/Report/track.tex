\section{Tracking}

In this section we will discuss the main module of the project which is the tracker. A simple Kalman filter tracker has been provided in the tracking framework. It uses results of the \textbf{Representer} to track the position and the extent of the object being tracked. The Kalman filter works by estimating an unobservable state which is updated in time with a linear state update and additive Gaussian noise. It uses the technique described in \cite{Arulampalam01atutorial}.

The basic Kalman filter provided in the framework estimates the position and the extent of the target. We also included the velocity and did the necessary modifications. Now the initializations for the Kalman filter is done as shown in the following snippet:

\begin{verbatim}

Tracker.H          = eye(6);        % System model
Tracker.Q          = 0.1 * eye(6);  % System noise
traker_state       = eye(6);        % Measurement model
traker_state(1,3)  = 1;
traker_state(2,4)  = 1;
Tracker.F          = traker_state;
Tracker.R          = 5 * eye(6);    % Measurement noise
Tracker.innovation = 0;
Tracker.track      = @multiple_kalman_step2;
\end{verbatim}

Our new measurements obtained from the \textbf{Representer} are compounded by the centroid of the object, its velocity and the blob height and width as follows:

\begin{equation} m_{k} = [x_{k}, y_{k}, \dot{x}_{k}, \dot{y}_{k}, h_{k}, w_{k}] \end{equation}

\begin{itemize}
	\item $x_{t} and y_{t}$ are the position in both x and y coordinates.
	\item $\dot{x}_{t} and \dot{y}_{t}$ are the velocities in each component.
	\item $h_{t} and w_{t}$ are the height and the with of the bounding-box of the object.
\end{itemize}

Now, since we take into account the velocity as a measurement in Kalman filtering, we also have changed the transition model of the filter. F is no longer an identity matrix, but adds the velocity in each position compound. The new composition of the measurements and the transition model makes Kalman filter be able to predict also the size of the new blob. This makes a lot of sense since it can be useful to assume precisely which is the corresponding blob in the next frame. Moreover, in order to improve more the tracker, we played with the internal parameters of the Kalman filter. Those are the \textit{System Noise} \textbf{K.Q} and the \textit{Measurement Noise} \textbf{K.R}. 

Regarding the \textit{System Noise} \textit{K.Q} of the filtering, we can observe that giving it higher values the tracker follows faster the \textit{Representer} blob. This is because it makes increase the \textit{Kalman Gain} which is the believing in \textit{innovation} of the filter. The \textit{innovation} is the difference between the new real measurement of the \textbf{Representer} and the predicted - a priori - state. So, the new prediction - a posteriori - which is the prediction from the state before plus the Kalman trust in the innovation, gives more closely measurement approximation to the real one. This could be very useful whether the measurements obtained are very trusty. Otherwise, if the detection - for any reason - is not able to obtain correct blobs, the tracker will follow fastly its wrong measurements. In those cases, if the \textit{System Noise} is small, we can smoothly follow the object.

On the other hand, the \textit{Measurement Noise K.R} with higher values the \textit{Kalman Prediction} trust less the measurements. \textit{K.R} is added to calculate the state covariance which its invert is used to obtain the \textit{Kalman Gain} - lower in this case -. As we have said before, the \textit{Kalman Gain} is the degree of believe on the \textit{innovation}. With low values, the measurements are less trusted, so the tracking follows smoothly the prediction. 

One of the highest problems we have in the tracking step is caused by the \textbf{Detector}. We are trying to track faces, but the faces that are found by the \textbf{Detector} are encountered inside a blob. When a face is found, the bounding-box and the centroid of the face are returned. But, when no face is found, the \textbf{Detector} returns the whole blob. If the \textbf{Representer} assures that this blob is the one where the face would have been found, it also returns the blob. This confuses the tracker, because the new centroid will be the centroid of the person instead of the face. Also happens the same with the size of the blob; the tracker is unable to predict correctly the size of the object. Even thought we were interested in face tracking, due to the problems genreated by our Face Recognizer, we have developed a new version of the tracker that takes only into account all the person, not the faces. When a face is recognized, the tracking is done over whole person.
%
